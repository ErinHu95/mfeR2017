% Created 2015-11-04 Wed 23:10
\documentclass[11pt]{article}
\usepackage[utf8]{inputenc}
\usepackage[T1]{fontenc}
\usepackage{fixltx2e}
\usepackage{graphicx}
\usepackage{longtable}
\usepackage{float}
\usepackage{wrapfig}
\usepackage{rotating}
\usepackage[normalem]{ulem}
\usepackage{amsmath}
\usepackage{textcomp}
\usepackage{marvosym}
\usepackage{wasysym}
\usepackage{amssymb,setspace}
\usepackage[colorlinks=true,linkcolor=blue,urlcolor=blue,citecolor=blue,anchorcolor=blue]{hyperref}
\tolerance=1000
\usepackage[hmargin=1in, vmargin = 1in]{geometry}
\author{Brett Dunn and Mahyar Kargar}
\date{\today}
\title{2017 R MFE Programming Workshop Lab 3}
\hypersetup{
  pdfkeywords={},
  pdfsubject={},
  pdfcreator={Emacs 24.5.1 (Org mode 8.2.10)}}
\usepackage{Sweave}
\begin{document}
\Sconcordance{concordance:Lab3.tex:Lab3.Rnw:%
1 26 1 1 0 70 1}


\maketitle
\onehalfspacing
\section{CAPM Failures}
\label{sec-1}
In this lab we are going to replicate some of basic results from Fama
and French's 1993 paper \emph{Common Risk Factors in the Returns of Stocks
and Bonds}. Kenneth French provides a phenomenal data library on his
\href{http://mba.tuck.dartmouth.edu/pages/faculty/ken.french/data_library.html}{website}. The dataset for this week contains two files:

\begin{itemize}
\item \emph{FFfactors.csv} contains returns of the famous Fama-French risk
factors $mkt.RF$ (the excess return on the market), $HML$, and $SMB$
along with the risk free rate $RF$.
\item \emph{FFports.csv} contains the returns of the 25 Fama-French
portfolios. I will denote the returns of these portfolios as
$R_{it}$ for $i=1,\ldots,25$.
\end{itemize}

Read in both of these datasets. First we will estimate the CAPM
$\beta$ for each of these 25 portfolios. You will likely need to clean
up the dates. Also, limit the data to be from January 1963 through the
end of 2013. The $\beta$ is estimated from the following time series
regression for each portfolio:

\begin{equation*}
R^{e}_{it} = \alpha_{i} + \beta_{i} mkt_{t} + \epsilon_{it} \ \ t=1,\ldots,T
\end{equation*}

$R^{e}_{it} = R_{it} - RF_{t}$ is the excess return on portfolio
$i$. Now calculate the average return for each portfolio over the
sample period. Plot the average return versus $\beta_{i}$ for all 25
portfolios. If the CAPM holds, then average return should linearly
increase in the $\beta_{i}$. Does this appear to be true?

\textbf{Note} doing the above will require a number of steps. To get you
 started, here are some hints:

\begin{itemize}
\item Use \verb~fread~ from \verb~data.table~ to create data.tables from the .csv files. 
\item You'll need to clean up the dates so that you can subset. Use
\verb~lubridate~.
\item You'll need to use a join or a merge. Before you merge, you will need to use \verb~melt~ from the package \verb~data.table~ to create a long table from \emph{FFfactors.csv} 
\item You'll need to run regressions on groups. Recall that you can put anything into j!
\end{itemize}

\section{Fama-French Model}
\label{sec-2}

Now we will look at the famous Fama-French 3 factor model. Instead of
estimating just $\beta_{i}$, estimate $\beta_{i}$, $h_{i}$, and
$s_{i}$ for each portfolio using the following time series regression:

\begin{equation*}
R^{e}_{it} = \alpha_{i} + \beta_{i} mkt.RF_{t} + h_{i} HML_{t} + s_{i} SMB_{t} + \epsilon_{it}, \ \ t=1,\ldots,T
\end{equation*}

Calculate the average returns for the 3 Fama-French factors
$E[mkt.RF_{t}]$, $E[HML_{t}]$, and $E[SMB_{t}]$. Now for each
portfolio, calculate the predicted value:

\begin{equation*}
pred_{i} = \beta_{i} \mathbb{E}[mkt.RF_{t}] + h_{i} \mathbb{E}[HML_{t}] + s_{i} \mathbb{E}[SMB_{t}]
\end{equation*}

Plot this predicted value versus the average excess return for each
portfolio. Do things look a little better?
% Emacs 24.5.1 (Org mode 8.2.10)
\end{document}
