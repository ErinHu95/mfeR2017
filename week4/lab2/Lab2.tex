\documentclass[12pt]{article}
\usepackage[utf8]{inputenc}
\usepackage[T1]{fontenc}
\usepackage{fixltx2e}
\usepackage{graphicx}
\usepackage{longtable}
\usepackage{float}
\usepackage{wrapfig}
\usepackage{rotating}
\usepackage[normalem]{ulem}
\usepackage{amsmath}
\usepackage{textcomp}
\usepackage{marvosym}
\usepackage{wasysym}
\usepackage{amssymb,setspace}
\usepackage[colorlinks=true,linkcolor=blue,urlcolor=blue,citecolor=blue,anchorcolor=blue]{hyperref}
\tolerance=1000
\usepackage[hmargin=1in, vmargin = 1in]{geometry}
\author{Brett Dunn and Mahyar Kargar}
\date{Fall 2017}
\title{2017 MFE R Programming Workshop\\
Lab 2}
\usepackage{Sweave}
\begin{document}
\Sconcordance{concordance:Lab2.tex:Lab2.Rnw:%
1 21 1 1 0 55 1}


\maketitle
\onehalfspacing
Getting started on this lab is likely to be a bit harder than the
previous lab. It will require you to do some reading of R package
manuals/vignettes. As you become more experienced with R programming,
this will be commonplace, so it is best to become comfortable with it
now. Having the skill of learning about a new package from its
standard documentation will be immensely useful. R is as useful as it
is a programming language because of the packages the community has
built around it. Learn to use them and you will quickly understand why
R is such a fantastic data analytics platform.

\section*{Interpolating the Yield Curve}
\label{sec-1}
Date on the treasury yield curve can be found \href{http://www.treasury.gov/resource-center/data-chart-center/interest-rates/Pages/TextView.aspx?data=yield}{here}.


\begin{itemize}
\item Download the data for the latest day and import it into R (the last
row on that page). Notice that the yields are for unequally spaced
intervals (1mo, 3mo, 1 year, etc.)
\item Using the lubridate and xts packages construct (you can use \verb~as.xts()~
or \verb~xts()~) an XTS object with two columns that has the dates and
the yields. For example on November 3, 2014 you would get:
\end{itemize}

\begin{center}
\begin{tabular}{lr}
Maturity Date & Yield\\
\hline
December 3, 2014 & 0.03\\
February 3, 2015 & 0.04\\
May 3, 2015 & 0.07\\
\ldots{} & \ldots{}\\
\end{tabular}
\end{center}

\begin{itemize}
\item Plot this yield curve (i.e. yield vs. maturity at a given date)
\item Now we are going to interpolate the missing yields (ie 2 months, 4
months, 5 months, etc.) all the way out to 30 years. Create an empty
xts object that has the monthly date index starting at the first
monthly observation going out to 30 years. The yields column will
have \verb~NA~ values to begin, but you will want to populate the rows
that you have observations for with the data from the treasury
website.
\item Now create two more columns in your dataset using \verb~na.approx()~ and
\verb~na.spline()~ from the xts package. These functions will replace the
\verb~NA~ values with interpolated values.
\item Plot your interpolated yield curves
\end{itemize}
\end{document}
