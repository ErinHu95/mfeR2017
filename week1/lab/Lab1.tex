\documentclass[12pt]{article}
\usepackage{setspace}
\usepackage{amsmath,amsthm}
\usepackage{amssymb}
\usepackage{graphicx}
\usepackage[hmargin=1in, vmargin = 1in]{geometry}
\usepackage{epsfig}
\usepackage{hyperref}
\usepackage{natbib}
\title{MFE R Programming Workshop Lab 1}
\date{October 04, 2017}
\author{Brett R. Dunn and Mahyar Kargar}
\begin{document}
\maketitle

\section{Call Options}
A call option on a stock $S$, maturing at time $T$, struck at $K$
is the ability, but not requirement, to purchase an asset $S$ at some
point $T$ at the price $K$. When a trader can borrow and lend as they like
at rate $r$ and the price of $S$ can be written at time $t$ as

\[
S_t = S_0 e^{\left(\mu-\frac{1}{2} \sigma^2 \right) t + \sigma B_t}
\]

where $B_t$ is normal with mean zero and variance $t$, and $\mu$ and
$\sigma$ are some fixed values.

Let $\Phi(z) = \frac{1}{\sqrt{2 \pi}} \int_{-\infty}^z e^{-\frac{1}{2}
  x^2}dx$ denote the CDF of the standard normal distribution. The
famous Black-Scholes (\cite{black1973pricing},
\cite{merton1973theory}) formula (which you will be taught how to
derive in your derivatives class) says that a price of a call option
on S maturing in T struck at K is

\[
S_0 \times \Phi(d_1) - e^{-rT} K \times \Phi(d_2)
\]

where

\begin{eqnarray*}
  d1 = \frac{\log(S_0 / K) + \left(r + \frac{1}{2} \sigma^2 \right)
    T}{\sigma \sqrt{T}} \\
  d2 = \frac{\log(S_0 / K) + \left(r - \frac{1}{2} \sigma^2 \right)
    T}{\sigma \sqrt{T}}
\end{eqnarray*}

\section{Questions}

\begin{itemize}
\item Write an R function that takes parameters $r$, $T$, $K$,
  $S_0$, and $\sigma$ and computes the Black-Scholes call price. You
  will somehow need to evaluate $\Phi(x)$. There are a couple of ways
  to do this, but all should start with considering how to find the
  appropriate function.
\item Evaluate the above function for the values $T=1$, $r=.04$,
  $\sigma=.25$, $K=95$, and $S_0 = 100$.
\item It may be that we need to do this for many parameters.  Compute
  what the price of a call maturing in $T = 1$ year should be on a
  stock with current price $S_0 = 100$ and volatility $\sigma = .2$,
  when the riskless rate of interest is $r = .05$. Write code to do
  this for every strike $K \in \{75, 76, 77, \ldots, 124, 125\}$, and
  print the results to the screen.  Now suppose that you want to do
  this for stocks of different maturities also, and that you need to
  use these prices to conduct some further analysis. For the same
  $S_0$, $r$, and $\sigma$, populate a matrix with the prices of an
  option for strikes and maturities $(K,T) \in \{75, 76, 77, \ldots,
  124, 125\} \times \{1/12, 2/12, \ldots 23/12, 2\}$. Again compare
  your results with the results from the built-in R function.
\item The file \emph{optionsdata.csv} contains the parameters for various
options. Read in this file and compute the Black-Scholes price for
these options. Append a column to the dataset and output the results
to its own csv file.
\end{itemize}

\bibliographystyle{jf} \bibliography{refs}

\end{document}


