\documentclass[12pt]{article}
\usepackage{setspace}
\usepackage{amsmath,amsthm}
\usepackage{amssymb}
\usepackage{graphicx}
\usepackage[hmargin=1in, vmargin = 1in]{geometry}
\usepackage{epsfig}
\usepackage[colorlinks=true,linkcolor=blue,urlcolor=blue,citecolor=blue,anchorcolor=blue]{hyperref}
\usepackage{natbib}
\title{MFE R Programming Workshop Lab 1}
\date{\today}
\author{Brett R. Dunn and Mahyar Kargar}
\begin{document}
\maketitle
\onehalfspacing
\section*{Call Options}
A European call option on a stock with price $S$, expiration date $T$, and strike price $K$
is the right, but not obligation, to buy the underlying asset at time $T$ at price $K$. 
If the stock price dynamics follow a geometric Brownian motion, it can be shown that the stock price at time $t$, $S_t$, can be written as

\[
S_t = S_0 e^{\left(\mu-\frac{1}{2} \sigma^2 \right) t + \sigma B_t}
\]

where $S_0$ is the currenct stock price, $B_t$ is a standard Brownian motion which is normally distributed with mean zero and variance $t$, and $\mu$ and
$\sigma$ (expected return and volatility of the stock return) are some fixed values.

Let $N(z) = \frac{1}{\sqrt{2 \pi}} \int_{-\infty}^z e^{-\frac{1}{2}
  x^2}dx$ denote the CDF of the standard normal distribution. The
famous Black-Scholes \citep{black1973pricing,merton1973theory},
formula (which you will learn how to
derive in your derivatives class) says that a price of a call option
on a stock with price $S$ maturing at time $T$ with strike price $K$ is

\[
S_0 \times N(d_1) - e^{-rT} K \times N(d_2)
\]

where $r$ is the risk-free rate, and

\begin{align*}
  d_1 &\equiv \frac{\ln(S_0 / K) + \left(r + \frac{1}{2} \sigma^2 \right)
    T}{\sigma \sqrt{T}} \\
  d_2 &\equiv \frac{\ln(S_0 / K) + \left(r - \frac{1}{2} \sigma^2 \right)
    T}{\sigma \sqrt{T}} = d_1-\sigma\sqrt{T}
\end{align*}

\section*{Questions}

\begin{enumerate}
\item Write an R function that takes parameters $r$, $T$, $K$,
  $S_0$, and $\sigma$ and computes the Black-Scholes call price. You
  will somehow need to evaluate $N(x)$. There are a couple of ways
  to do this, but all should start with considering how to find the
  appropriate function.
\item Evaluate the above function for the values $T=1$, $r=.04$,
  $\sigma=.25$, $K=95$, and $S_0 = 100$. You can compare your answer
  to the one obtained using the \texttt{Black\_Scholes()} function from the \texttt{qrmtools} package. 
\item We may need to find the price of a call option for many parameters.  Compute
  what the price of a call maturing in $T = 1$ year should be on a
  stock with current price $S_0 = 100$ and volatility $\sigma = .2$,
  when the riskless rate of interest is $r = .05$. Write code to do
  this for every strike $K \in \{75, 76, 77, \ldots, 124, 125\}$, and
  print the results on R console.  Now suppose that you want to do
  this for stocks of different maturities also, to
  use these call prices to conduct some further analysis. For the same
  $S_0$, $r$, and $\sigma$, populate a matrix with the prices of an
  option for strikes and maturities $(K,T) \in \{75, 76, 77, \ldots,
  124, 125\} \times \{1/12, 2/12, \ldots 23/12, 2\}$. Again compare
  your results with the results using the \texttt{Black\_Scholes()} function.
\item The file \emph{optionsdata.csv} contains the parameters for various
options. Read in this file and compute the Black-Scholes price for
these options. Append a column to the dataset for the call price and write the results
to its own csv file.
\end{enumerate}
\newpage
\bibliographystyle{jf} \bibliography{refs}

\end{document}


